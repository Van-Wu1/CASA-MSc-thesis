% Options for packages loaded elsewhere
\PassOptionsToPackage{unicode}{hyperref}
\PassOptionsToPackage{hyphens}{url}
\documentclass[
  12pt,
  oneside]{book}
\usepackage{xcolor}
\usepackage[left=4cm, right=3cm, top=2.5cm, bottom=2.5cm]{geometry}
\usepackage{amsmath,amssymb}
\setcounter{secnumdepth}{5}
\usepackage{iftex}
\ifPDFTeX
  \usepackage[T1]{fontenc}
  \usepackage[utf8]{inputenc}
  \usepackage{textcomp} % provide euro and other symbols
\else % if luatex or xetex
  \usepackage{unicode-math} % this also loads fontspec
  \defaultfontfeatures{Scale=MatchLowercase}
  \defaultfontfeatures[\rmfamily]{Ligatures=TeX,Scale=1}
\fi
\usepackage{lmodern}
\ifPDFTeX\else
  % xetex/luatex font selection
\fi
% Use upquote if available, for straight quotes in verbatim environments
\IfFileExists{upquote.sty}{\usepackage{upquote}}{}
\IfFileExists{microtype.sty}{% use microtype if available
  \usepackage[]{microtype}
  \UseMicrotypeSet[protrusion]{basicmath} % disable protrusion for tt fonts
}{}
\usepackage{setspace}
\makeatletter
\@ifundefined{KOMAClassName}{% if non-KOMA class
  \IfFileExists{parskip.sty}{%
    \usepackage{parskip}
  }{% else
    \setlength{\parindent}{0pt}
    \setlength{\parskip}{6pt plus 2pt minus 1pt}}
}{% if KOMA class
  \KOMAoptions{parskip=half}}
\makeatother
\usepackage{color}
\usepackage{fancyvrb}
\newcommand{\VerbBar}{|}
\newcommand{\VERB}{\Verb[commandchars=\\\{\}]}
\DefineVerbatimEnvironment{Highlighting}{Verbatim}{commandchars=\\\{\}}
% Add ',fontsize=\small' for more characters per line
\usepackage{framed}
\definecolor{shadecolor}{RGB}{248,248,248}
\newenvironment{Shaded}{\begin{snugshade}}{\end{snugshade}}
\newcommand{\AlertTok}[1]{\textcolor[rgb]{0.94,0.16,0.16}{#1}}
\newcommand{\AnnotationTok}[1]{\textcolor[rgb]{0.56,0.35,0.01}{\textbf{\textit{#1}}}}
\newcommand{\AttributeTok}[1]{\textcolor[rgb]{0.13,0.29,0.53}{#1}}
\newcommand{\BaseNTok}[1]{\textcolor[rgb]{0.00,0.00,0.81}{#1}}
\newcommand{\BuiltInTok}[1]{#1}
\newcommand{\CharTok}[1]{\textcolor[rgb]{0.31,0.60,0.02}{#1}}
\newcommand{\CommentTok}[1]{\textcolor[rgb]{0.56,0.35,0.01}{\textit{#1}}}
\newcommand{\CommentVarTok}[1]{\textcolor[rgb]{0.56,0.35,0.01}{\textbf{\textit{#1}}}}
\newcommand{\ConstantTok}[1]{\textcolor[rgb]{0.56,0.35,0.01}{#1}}
\newcommand{\ControlFlowTok}[1]{\textcolor[rgb]{0.13,0.29,0.53}{\textbf{#1}}}
\newcommand{\DataTypeTok}[1]{\textcolor[rgb]{0.13,0.29,0.53}{#1}}
\newcommand{\DecValTok}[1]{\textcolor[rgb]{0.00,0.00,0.81}{#1}}
\newcommand{\DocumentationTok}[1]{\textcolor[rgb]{0.56,0.35,0.01}{\textbf{\textit{#1}}}}
\newcommand{\ErrorTok}[1]{\textcolor[rgb]{0.64,0.00,0.00}{\textbf{#1}}}
\newcommand{\ExtensionTok}[1]{#1}
\newcommand{\FloatTok}[1]{\textcolor[rgb]{0.00,0.00,0.81}{#1}}
\newcommand{\FunctionTok}[1]{\textcolor[rgb]{0.13,0.29,0.53}{\textbf{#1}}}
\newcommand{\ImportTok}[1]{#1}
\newcommand{\InformationTok}[1]{\textcolor[rgb]{0.56,0.35,0.01}{\textbf{\textit{#1}}}}
\newcommand{\KeywordTok}[1]{\textcolor[rgb]{0.13,0.29,0.53}{\textbf{#1}}}
\newcommand{\NormalTok}[1]{#1}
\newcommand{\OperatorTok}[1]{\textcolor[rgb]{0.81,0.36,0.00}{\textbf{#1}}}
\newcommand{\OtherTok}[1]{\textcolor[rgb]{0.56,0.35,0.01}{#1}}
\newcommand{\PreprocessorTok}[1]{\textcolor[rgb]{0.56,0.35,0.01}{\textit{#1}}}
\newcommand{\RegionMarkerTok}[1]{#1}
\newcommand{\SpecialCharTok}[1]{\textcolor[rgb]{0.81,0.36,0.00}{\textbf{#1}}}
\newcommand{\SpecialStringTok}[1]{\textcolor[rgb]{0.31,0.60,0.02}{#1}}
\newcommand{\StringTok}[1]{\textcolor[rgb]{0.31,0.60,0.02}{#1}}
\newcommand{\VariableTok}[1]{\textcolor[rgb]{0.00,0.00,0.00}{#1}}
\newcommand{\VerbatimStringTok}[1]{\textcolor[rgb]{0.31,0.60,0.02}{#1}}
\newcommand{\WarningTok}[1]{\textcolor[rgb]{0.56,0.35,0.01}{\textbf{\textit{#1}}}}
\usepackage{longtable,booktabs,array}
\usepackage{calc} % for calculating minipage widths
% Correct order of tables after \paragraph or \subparagraph
\usepackage{etoolbox}
\makeatletter
\patchcmd\longtable{\par}{\if@noskipsec\mbox{}\fi\par}{}{}
\makeatother
% Allow footnotes in longtable head/foot
\IfFileExists{footnotehyper.sty}{\usepackage{footnotehyper}}{\usepackage{footnote}}
\makesavenoteenv{longtable}
\usepackage{graphicx}
\makeatletter
\newsavebox\pandoc@box
\newcommand*\pandocbounded[1]{% scales image to fit in text height/width
  \sbox\pandoc@box{#1}%
  \Gscale@div\@tempa{\textheight}{\dimexpr\ht\pandoc@box+\dp\pandoc@box\relax}%
  \Gscale@div\@tempb{\linewidth}{\wd\pandoc@box}%
  \ifdim\@tempb\p@<\@tempa\p@\let\@tempa\@tempb\fi% select the smaller of both
  \ifdim\@tempa\p@<\p@\scalebox{\@tempa}{\usebox\pandoc@box}%
  \else\usebox{\pandoc@box}%
  \fi%
}
% Set default figure placement to htbp
\def\fps@figure{htbp}
\makeatother
\setlength{\emergencystretch}{3em} % prevent overfull lines
\providecommand{\tightlist}{%
  \setlength{\itemsep}{0pt}\setlength{\parskip}{0pt}}
\usepackage[style=apa,]{biblatex}
\addbibresource{book.bib}
\addbibresource{packages.bib}
\addbibresource{test.bib}
\usepackage[none]{hyphenat}
\pagestyle{plain}
\raggedbottom
\usepackage[nottoc,notlot,notlof]{tocbibind}
\usepackage{longtable}
\usepackage{pdfpages}
\usepackage[width=\textwidth]{caption}

\usepackage{fancyhdr}
\pagestyle{fancy}
\fancyhf{}
\setlength{\headheight}{15pt}%
\fancyhead[RO,RE]{\nouppercase{\leftmark}}
\fancyfoot[CO, CE] {\thepage}
\renewcommand{\headrulewidth}{0pt}
\renewcommand{\footrulewidth}{0pt}
\usepackage{booktabs}
\usepackage{longtable}
\usepackage{array}
\usepackage{multirow}
\usepackage{wrapfig}
\usepackage{float}
\usepackage{colortbl}
\usepackage{pdflscape}
\usepackage{tabu}
\usepackage{threeparttable}
\usepackage{threeparttablex}
\usepackage[normalem]{ulem}
\usepackage{makecell}
\usepackage{xcolor}
\usepackage{bookmark}
\IfFileExists{xurl.sty}{\usepackage{xurl}}{} % add URL line breaks if available
\urlstyle{same}
\hypersetup{
  pdftitle={CASA dissertation using Bookdown},
  hidelinks,
  pdfcreator={LaTeX via pandoc}}

\title{CASA dissertation using Bookdown}
\author{Yifan Wu\\
\strut \\
CASA0010, MSc Smart Cities Dissertation\\
\strut \\
Supervisor: Dr Duncan Smith\\
\strut \\
Repository: \url{https://andrewmaclachlan.github.io/CASA0005repo/}\\
\strut \\
This dissertation is submitted in part requirement for the\\
MSc (Or MRes) in the Centre for Advanced Spatial Analysis,\\
Bartlett Faculty of the Built Environment, UCL\\
\strut \\
Word count: unknown 17:08 update}
\date{2025-08-24}

\begin{document}
\maketitle

\setstretch{1.5}
\pagenumbering{roman}

\chapter*{Abstract}\label{abstract}

Some abstract text

\pagenumbering{roman}

\chapter*{Declaration}\label{declaration}

I hereby declare that this dissertation is all my own original work and that all sources have been acknowledged. It is xxx words in length

\chapter*{Acknowledgements}\label{acknowledgements}

I would like to thank blah blah

% Trigger ToC creation in LaTeX
\setcounter{tocdepth}{3}
\tableofcontents
\listoffigures
\listoftables

\chapter*{Abbreviations}\label{abbreviations}

\begin{table}
\centering
\begin{tabular}{ll}
\toprule
\textbf{Term} & \textbf{Abbreviation}\\
\midrule
Digital Elevation Model & DEM\\
Digital Surface Model & DSM\\
Digital Terrain Model & DTM\\
\bottomrule
\end{tabular}
\end{table}

\chapter{Bookdown basics}\label{bookdown-basics}

\pagenumbering{arabic} 

Bookdown enables us to take raw text files (e.g.~Rmarkdown files) and output them into a number of different formats with ease. This is more useful than LaTex as you can create a word version for comments from your supervisor, a pdf for final submission and an online book for your own portfolio.

\section{Structure}\label{structure}

A bookdown book simply combines multiple Rmarkdown files into .pdf, html or .epub (but i've disabled epub here).

All Rmakdown files must be located in the base (or root) of the project. For example, don't got putting the .Rmd in a folder called chapters and then wonder why it's not working. They all must be in the same folder as the project file. You can however put output figure images into a folder (e.g.~figures) then call them in.

\section{Building the book}\label{building-the-book}

You will need the packages: \texttt{bookdown,\ kabble,\ knitr}

You also need to install \texttt{tinytex::install\_tinytex()} for generating .pdfs from your book.

Once installed you can build the book by clicking the Build Book icon under the build tab in the top right quadrant.

To export the book to word follow the code below from \href{https://www.eddjberry.com/post/writing-your-thesis-with-bookdown/}{Edd Berry}:

\begin{Shaded}
\begin{Highlighting}[]
\NormalTok{bookdown}\SpecialCharTok{::}\FunctionTok{preview\_chapter}\NormalTok{(}\StringTok{"01{-}intro.Rmd"}\NormalTok{,}
                \AttributeTok{output\_format =} \StringTok{"bookdown::word\_document2"}\NormalTok{,}
                \AttributeTok{output\_file =} \FunctionTok{paste0}\NormalTok{(}\StringTok{"thesis{-}intro{-}"}\NormalTok{, }
                                     \FunctionTok{format}\NormalTok{(}\FunctionTok{Sys.Date}\NormalTok{(), }
\NormalTok{                                            (}\StringTok{"\%d{-}\%m{-}\%y"}\NormalTok{)), }
                                     \StringTok{".docx"}\NormalTok{),}
                \AttributeTok{output\_dir =} \StringTok{"chapter{-}previews"}\NormalTok{)}
\end{Highlighting}
\end{Shaded}

You can also setup a reference document if you wish that word will style it on. To do so:

\begin{enumerate}
\def\labelenumi{\arabic{enumi}.}
\tightlist
\item
  export to word using the code above.
\item
  change the headings styles in word
\item
  save the document
\item
  add the \texttt{output\_options()} function to the code above
\end{enumerate}

See \href{https://bookdown.org/yihui/rmarkdown-cookbook/word-template.html}{Custom Word Templates} for more detailed instructions.

\begin{Shaded}
\begin{Highlighting}[]
\NormalTok{bookdown}\SpecialCharTok{::}\FunctionTok{preview\_chapter}\NormalTok{(}\StringTok{"01{-}intro.Rmd"}\NormalTok{,}
                \AttributeTok{output\_format =} \StringTok{"bookdown::word\_document2"}\NormalTok{,}
                \AttributeTok{output\_file =} \FunctionTok{paste0}\NormalTok{(}\StringTok{"thesis{-}intro{-}"}\NormalTok{, }
                                     \FunctionTok{format}\NormalTok{(}\FunctionTok{Sys.Date}\NormalTok{(), }
\NormalTok{                                            (}\StringTok{"\%d{-}\%m{-}\%y"}\NormalTok{)), }
                                     \StringTok{".docx"}\NormalTok{),}
                \AttributeTok{output\_dir =} \StringTok{"chapter{-}previews"}\NormalTok{,}
                \AttributeTok{output\_options =} \FunctionTok{list}\NormalTok{(}\AttributeTok{reference\_docx =} \StringTok{"word{-}style{-}ref.docx"}\NormalTok{))}
\end{Highlighting}
\end{Shaded}

You can also export the whole document as word by clicking Build Book and selecting it. However, note that tables built with \texttt{Kabble} don't work with word, so you have the following options

\begin{itemize}
\tightlist
\item
  create the html (gitbook) and copy the tables into word
\item
  either exclude them (\texttt{eval=FALSE}) in the code chunk header
\item
  use .pdf.
\end{itemize}

\section{YAMLs}\label{yamls}

You will see some files with a \texttt{.yml} extension. These stand for yet another markup language.

Open the \texttt{\_output.yml} and \texttt{\_bookdown.yml}.

The \texttt{\_output.yml} controls the settings for each outputted format.

The \texttt{\_bookdown.yml} controls the order in which the files are made into chapters - you can also change the \texttt{Chapter} title if you wish here.

I have set these up to create nice outputs, so there is really no need to change anything, unless you want to add a new chapter and include it in the list. To do so, simply create a new Rmarkdown document, delete all the default content, add a first level heading, and the then add it to where you want it to appear in the document in the \texttt{\_bookdown.yml}.

\section{Formatting}\label{formatting}

You'll notice in the \texttt{\_output.yml} that the various formats (html (or gitbook) and pdf) are calling styling codes. The gitbook calls \texttt{style.css} and the \texttt{pdf\_book} calls \texttt{preamble.tex}. These files just style the various outputs.

I have copied in a basic styling with a CASA logo in the table of contents, you can replace this with other variations of the file if you wish or look at the \href{https://github.com/rstudio/bookdown-demo}{minimal bookdown example} or my \href{https://github.com/andrewmaclachlan/CASA0005repo/blob/master/assets/style.css}{CASA0005 .css file}.

The best place to learn more about styling is the \href{https://rstudio4edu.github.io/rstudio4edu-book/intro-bookdown.html}{RStudio for education bookdown guide}

For the .pdf version in the \texttt{preamble.tex} we have to use LaTex code. I am by no means an expert in this. If you Google each package it will tell you what it does. There isn't much here really, but all the header stuff refers to the headings at the top of each page.

This tutorial from OurCoding Club explains some of the LaTex packages in a bit more detail: \url{https://ourcodingclub.github.io/tutorials/rmarkdown-dissertation/}

\section{Index file}\label{index-file}

Open up the \texttt{index.Rmd} this \texttt{Rmd} must be called first in the \texttt{\_bookdown.yml} list. There are a number of options here, but i've set them all up to be compliant with the CASA thesis requirements.

You will need to change your title and name etc. Make sure you leave the a space here: \texttt{\textbar{}\ Andrew\ MacLachlan} this won't work \texttt{\textbar{}Andrew\ MacLachlan}

If you were to be printing your work, you'd want to change the \texttt{classoption} to \texttt{twosides} and make sure the \texttt{geometry\ margins} were correct. \texttt{linestrech} refers to the line spacing and the bibliography stuff we come on to later.

Input your GitHub repo and add a description.

If you ever wanted to just create a report and not a thesis you can change the \texttt{documentclass} to other options such \texttt{report}, \texttt{article} or \texttt{letter}

\section{Preamble}\label{preamble}

Open the \texttt{preable.Rmd} and you will see all the sections that are required before the main text (e.g.~Declaration and so on). At the top of the page i've used a code chunk set to LaTex, saying to use Roman numbering as we don't want page 1 to be the Declaration, we want it to be the first page of the Introduction. There are two conditions for each of the sections that state if output to HTML (gitbook) then do this, if output to LaTex then do this. This is the only place we have this. In our bookdown HTML we want to be able to click these sections, but in our LaTex .pdf we don't want them to appear in the table of contents. This is what this code is doing.

The Abstract is on the \texttt{index.Rmd} the same code condition applies to it, with Roman numbering also specified.

If you look back in the \texttt{\_output.yml} you'll also see the \texttt{toc} (table of contents) is set to false. By default this appears right after the title, but we want this to come after our \texttt{preable.Rmd}. You'll see that i've called \texttt{\textbackslash{}tableofcontents}, \texttt{\textbackslash{}listoffigures} and \texttt{\textbackslash{}listoftables} in the correct place again using a LaTex code chunk. These aren't required for the bookdown output.

The last section here is the abbreviations. To make this really easy, i've created an excel document to add them into. The code here will load that and then use the \texttt{kable} package to make a table. More on this later.I've also done the same for the research log in the appendix --- excel sheet called \texttt{research\_log}.

\section{Change this to a thesis}\label{change-this-to-a-thesis}

Easy. Just change all the titles to what you want (e.g.~Introduction, Literature Review, Methodology, Discussion, Conclusion). Some of the latter \texttt{.Rmds} (Discussion etc) are ready to go!

\section{Word count}\label{word-count}

To get a word count install and then use the \href{https://github.com/benmarwick/wordcountaddin}{word count addin package} through Tools\textgreater Addins\textgreater Wordcount

\section{Adding a pdf}\label{adding-a-pdf}

If you wish to add another .pdf as an Appendix (in your .pdf) then again we need a bit of LaTex code

\texttt{\textbackslash{}includepdf{[}pages=\{-\}{]}\{mypdf.pdf\}}

If you look in the \texttt{08-Appendix.Rmd} then you will see another if LaTex section, simply add in the line of code above, replacing \texttt{mypdf.pdf} with your pdf title in the main project folder. It will then be appended to the thesis. Of course, this isn't required in the online book, but you just link to them on GitHub or embed .pdfs using:

You'd need a condition around this like in the \texttt{preamble.Rmd} but a link is fine. I embedded a .pdf in \href{https://andrewmaclachlan.github.io/CASA0005repo/assignment-resources-1.html}{the assignment resources of CASA0005}

\section{Writing code}\label{writing-code}

Use one project for your thesis and another for your analysis. Don't try and do it all in a thesis project. You can set your output folder from your main analysis project to the thesis project and then easily load the figures in.

\section{Package reproducibility}\label{package-reproducibility}

Have you ever created an R script, come back to it 6 months later and wonder why it's not working correctly? It's probably because of package updates.

\texttt{renv} (pronounced R - env) can capture the packages used in your project and re-create your current library. You simply:

\begin{enumerate}
\def\labelenumi{\arabic{enumi}.}
\tightlist
\item
  Create a new project - \texttt{renv::init()}
\item
  Create a snapshot - \texttt{renv::snapshot()}
\item
  Call the snapshot to load - \texttt{renv::restore()}
\end{enumerate}

The package information and dependencies are stored in a \texttt{renv.lock} file.

When R loads a package it gets it from the library path, which is where the packages live. Sometimes there are two libraries a system and a user library - use \texttt{.libPaths()}. The system library = the packages with R, the user library = packages you have installed.

When you load a package it loads the first instance it comes across, user comes before system. To check - \texttt{find.package("tidyverse")}

All your projects use these paths! If you load different packages and versions of them + dependencies. E.g.

\begin{itemize}
\tightlist
\item
  Project 1 used \texttt{sf} version 0.9-8
\item
  Project 2 used \texttt{sf} version 0.9-6
\end{itemize}

Switching between projects would mean you have the wrong version as they use the same libraries.

\texttt{renv} - each project gets it's own library! Project local libraries.

When you use \texttt{renv::init()} the library path will be changed to a project local one.

It will create a lock file that holds all the package information.

To re-create my environment once you have forked and pulled this repository you would use \texttt{renv::restore()}.

Of coruse some projects use the same package version --- such as \texttt{tidyverse}, \texttt{renv} has a global cache of all the libraries. So there is a massive database of your libraries then each project library links it from there, meaning you don't have 10 versions of the same \texttt{tidyverse}.

You can also interact with \texttt{git}

\begin{itemize}
\tightlist
\item
  \texttt{renv::history()} --- finds the commits where the lock file changed
\item
  \texttt{renv::revert(commit\ =\ "id")} --- changes the lock file back to what it was at a commit
\end{itemize}

For more information watch renv: Project Environments for R introduction video: \url{https://www.rstudio.com/resources/rstudioconf-2020/renv-project-environments-for-r/}

\chapter{Literature Review}\label{literature-review}

\section{Introduction to Urban Cycling Environment Assessment}\label{introduction-to-urban-cycling-environment-assessment}

Cycling has increasingly been considered a key component of sustainable urban transport systems because it is a low‑emission, space‑efficient mode that can deliver environmental, health and social co‑benefits (Yanocha \&\,Mawdsley, 2022). Evidence from international studies suggests that shifting a substantial share of trips to cycling can reduce greenhouse‑gas emissions, air pollutants and traffic externalities (Yanocha \&\,Mawdsley, 2022). Estimates from the World Health Organization's Health Economic Assessment Tool indicate that each kilometre cycled yields a societal benefit of about €0.16, whereas each kilometre driven imposes a cost of roughly €0.15 (Yanocha \&\,Mawdsley, 2022).

Research on the health impacts of cycling also suggests significant gains. A large Danish cohort study of more than 52,000 adults followed for 13 years found that habitual cycling was associated with lower risks of cardiovascular disease, respiratory disease, diabetes and all‑cause mortality, and these benefits were not materially diminished by exposure to traffic‑related air pollution (Logan et\,al., 2023). Modelling studies have further shown that, even in highly polluted cities, the health benefits of regular cycling generally outweigh the risks associated with increased inhalation of pollutants (Logan et\,al., 2023). In the Netherlands, a country with extensive cycling infrastructure, approximately 27 \% of all trips are made by bicycle; Fishman et\,al.~quantified the health benefits of this cycling culture at roughly 6,500 deaths prevented per year and about half a year of additional life expectancy per person (Fishman,\,Schepers \&\,Kamphuis, 2015).

Despite these findings, cycling participation and infrastructure quality vary widely between cities. Transport for London's international benchmarking study reported that bicycle mode share is about 1 \% in New York City, around 2 \% in London, and approximately 40 \% in Amsterdam (Transport for London, 2013). This heterogeneity highlights the need for systematic assessments of urban cycling environments to inform targeted investments. The FLOW project, an EU‑funded initiative, argues that improvements in walking and cycling infrastructure constitute some of the most promising long‑term measures for easing congestion because they are relatively inexpensive and can encourage shifts away from car use (Koska \&\,Rudolph, 2016). Surveys conducted for the same project reveal that experts recognise the potential of walking and cycling measures to reduce congestion but note that such measures are implemented infrequently, indicating an implementation gap (Koska \&\,Rudolph, 2016).

Investments in cycling infrastructure also appear to produce wider societal benefits. Research by the Institute for Transportation and Development Policy reports that separated cycle lanes can reduce cyclist injuries and fatalities even as ridership increases, and that improvements in air quality from modal shifts can lead to further reductions in premature deaths (Yanocha \&\,Mawdsley, 2022). In Washington, DC, analysis of the Capital Bikeshare system found that the presence of bikeshare stations was associated with a 2--3 \% reduction in traffic congestion on nearby roads (Wichman, 2016). Such results suggest that well‑designed cycling interventions can contribute not only to individual wellbeing but also to urban efficiency and economic productivity.

Overall, the literature indicates that cycling can play a meaningful role in reducing environmental impacts, improving public health and enhancing urban liveability. However, the marked disparities in cycling uptake and infrastructure provision across cities underscore the importance of context‑specific assessments. Rigorous evaluation of existing cycling environments and careful identification of infrastructural gaps are necessary to design effective policies and investments capable of realising the potential benefits of urban cycling.

\section{Review of Existing Evaluation Frameworks}\label{review-of-existing-evaluation-frameworks}

\subsection{Structural Rideability}\label{structural-rideability}

A critical dimension of cycling environment assessment is Structural Rideability, capturing physical infrastructure attributes that directly influence cycling comfort, safety, and user inclusivity. This concept aligns with established frameworks such as the Canadian Bikeway Comfort and Safety (Can-BICS) Classification System, which systematically evaluates cycling infrastructure based on safety performance and user comfort across different facility types (Ferster et al., 2023). Similarly, the Dutch CROW Design Manual emphasizes five key principles for effective cycling infrastructure: cohesion, directness, safety, comfort, and attractiveness (CROW, 2016), reinforcing the multidimensional nature of structural cycling environment assessment.

A foundational metric within this domain is the Level of Traffic Stress (LTS) (Mekuria et al., 2012), which categorizes road segments by traffic conditions, lane width, speed, and cycling infrastructure presence. Recent developments have extended LTS applicability through open-source mapping platforms, with Wasserman et al.~(2019) demonstrating that OpenStreetMap-derived LTS scores achieve 89.9\% accuracy compared to field-validated assessments, facilitating scalable urban mapping applications.

While LTS provides a robust framework for cycling infrastructure assessment, empirical research has identified opportunities for refinement and expansion of its core assessment dimensions. Recent studies have highlighted the critical importance of specific infrastructure characteristics that form the foundation of comprehensive cycling environment evaluation:

\begin{enumerate}
\def\labelenumi{\arabic{enumi}.}
\tightlist
\item
  Physical Infrastructure and Separation
\end{enumerate}

The design and quality of cycling infrastructure significantly influence safety outcomes and user comfort. Research consistently demonstrates that protected bike lanes and physically separated cycle tracks substantially reduce collision risk, with studies reporting injury rate reductions of up to 50\% compared to conventional on-road facilities (Harris et al., 2013; Reynolds et al., 2009). The type of physical separation---ranging from painted lanes to grade-separated infrastructure---creates varying degrees of perceived and actual safety, directly influencing cycling participation across different user groups.

\begin{enumerate}
\def\labelenumi{\arabic{enumi}.}
\setcounter{enumi}{1}
\tightlist
\item
  Traffic Speed and Volume Characteristics
\end{enumerate}

Vehicle speed and traffic volume represent fundamental determinants of cycling stress and safety. Studies indicate that slower traffic speeds are associated with significantly fewer cyclist injuries, with research demonstrating that combined bike infrastructure and traffic calming measures generate substantially higher cyclist comfort ratings (Sanders et al., 2021). The relationship between traffic characteristics and cycling safety forms a core component of stress-level assessment, with speed limits serving as critical design parameters for cycling infrastructure planning.

\begin{enumerate}
\def\labelenumi{\arabic{enumi}.}
\setcounter{enumi}{2}
\tightlist
\item
  Network Connectivity and Intersection Design
\end{enumerate}

The continuity of cycling networks and intersection treatments constitute essential elements of comprehensive cycling environment assessment. Research demonstrates that cyclist interactions become more severe and less safe at locations with cycling network discontinuities, highlighting the importance of seamless network connections. Intersection and crossing treatments are fundamental considerations in LTS evaluation, with studies showing that specialized intersection designs can significantly reduce cyclist-motorist conflicts and improve overall network usability.
4. Spatial Configuration and Design Standards

The geometric design of cycling facilities, including lane width, marking clarity, and spatial relationship to vehicular traffic, influences both objective and subjective safety measures. LTS assessment incorporates the number of lanes, effective vehicle speed, and the presence and type of bicycle facility, creating a comprehensive evaluation framework that accounts for the multidimensional nature of cycling infrastructure quality.

These infrastructure dimensions represent the core components of systematic cycling environment assessment, building upon established LTS principles while providing detailed evaluation criteria for evidence-based cycling infrastructure planning and prioritization.

\subsection{Environmental Perception}\label{environmental-perception}

Beyond structural attributes, environmental perception represents a critical dimension of cycling environment assessment, significantly influencing cycling comfort, route choice behavior, and overall cycling participation. This subjective yet quantifiable dimension encompasses multiple environmental factors that shape cyclists' psychological and physiological experiences during cycling activities.

\begin{enumerate}
\def\labelenumi{\arabic{enumi}.}
\item
  Visual Greenery and Aesthetic Quality
  The visual perception of greenery, measured through the Green View Index (GVI), constitutes a fundamental component of environmental cycling assessment. Studies examining cycling patterns demonstrate that eye-level greenness is positively associated with cycling frequency on both weekdays and weekends (Lu et al., 2020; Yang et al., 2023). Systematic reviews confirm that street greenery promotes active travel through the creation of visually attractive, safe, and comfortable environments (Gascon et al., 2024).
\item
  Air Quality and Pollution Exposure
\end{enumerate}

Nitrogen dioxide (NO₂) serves as a representative indicator of urban air pollution exposure for cyclists. NO₂ is predominantly transport-related, with most emissions from cars, trucks, and buses, directly reflecting traffic-related exposure conditions (Ma et al., 2024). As a regulated pollutant used to assess ambient air quality in urban environments, NO₂ provides a robust and standardized measure for cycling environment assessment. Studies demonstrate substantial spatial variations in NO₂ exposure along different cycling routes, with measurable implications for both physiological comfort and health safety perceptions (An et al., 2018).

\begin{enumerate}
\def\labelenumi{\arabic{enumi}.}
\setcounter{enumi}{2}
\tightlist
\item
  Natural Features and Landscape Elements
\end{enumerate}

The presence and accessibility of natural features---including parks, green spaces, and water bodies---constitute essential components of environmental perception in cycling assessment. Research consistently demonstrates that exposure to natural environments generates measurable psychological benefits, with studies showing that green spaces boost serotonin and dopamine levels in the brain, contributing to happiness and well-being (Lee \& Maheswaran, 2011).

Evidence indicates that people living in proximity to natural spaces have significantly improved mental health outcomes, with benefits persisting up to three years after establishing residence near greener areas (Nieuwenhuijsen et al., 2017). Furthermore, cross-national studies across 18 countries found that frequency of recreational visits to green, inland-blue, and coastal-blue spaces were all positively associated with well-being and negatively associated with mental distress (Hooyberg et al., 2021). The integration of natural features into cycling route evaluation reflects the understanding that proximity to lakes, parks, and natural landscapes enhances overall cycling experience through psychological restoration and mood improvement.

Collectively, these three environmental perception dimensions---visual greenery (GVI), air quality (NO₂), and natural features---provide a comprehensive framework for capturing the subjective yet quantifiable aspects of cycling environments that significantly influence user behavior, comfort, and participation decisions.

\subsection{Network Performance}\label{network-performance}

Urban cycling evaluations also rely extensively on network performance indicators such as connectivity, centrality, and infrastructure density. Network performance assessment focuses on evaluating the ease and convenience of movement within the network without necessarily specifying origin-destination pairs, thus offering a versatile tool for urban cycling assessments.

Contemporary approaches to network analysis have been revolutionized by open-source computational tools. Boeing (2017) developed OSMnx, a Python package that enables comprehensive street network analysis using OpenStreetMap data, allowing researchers to download, model, analyze, and visualize urban networks with unprecedented ease and accuracy. This methodological advancement has facilitated large-scale comparative studies of urban network structures across multiple cities and regions.

The theoretical foundations of urban network analysis were established through graph-theoretic approaches that emphasize topological properties. Porta et al.~(2006) introduced the primal graph methodology for urban street network analysis, demonstrating that centrality indices effectively capture the structural `skeleton' of urban areas. Their Multiple Centrality Assessment (MCA) framework provides a metric-based approach that investigates multiple centrality indices simultaneously, offering more comprehensive network evaluation than single-index approaches.

Recent research has expanded network performance assessment to incorporate cycling-specific infrastructure elements. Studies demonstrate that bicycle network connectivity significantly influences cycling behavior, with well-connected networks showing higher usage rates and broader demographic participation (Buehler \& Dill, 2016). Furthermore, the integration of bicycle parking facilities and connections to public transportation nodes represents essential components of comprehensive network performance evaluation, as these multimodal connections significantly enhance cycling network attractiveness and usability (Geurs et al., 2016).

Network density and structural coherence also play critical roles in cycling network effectiveness. Research indicates that cycling networks benefit from both high local connectivity and efficient long-distance connections, with network fragmentation representing a significant barrier to cycling adoption (Lowry et al., 2012). The application of graph-theoretic measures such as betweenness centrality and clustering coefficients provides quantitative frameworks for identifying critical network nodes and assessing overall network robustness for cycling infrastructure planning.

\section{Composite Index Approaches (CECI Models)}\label{composite-index-approaches-ceci-models}

Composite indices for assessing cycling environments have emerged as important tools for integrating multiple dimensions of urban cycling conditions into unified assessment metrics. Galarza-Torres et al.~(2020) developed an urban Bikeability Index (BI) to assess and prioritise bicycle infrastructure investments, addressing particularities of roads in urban contexts. Their methodology incorporates infrastructure quality, safety considerations, and accessibility factors to guide investment decisions for improved cyclist accessibility.

Hassanpour et al.~(2021) proposed a Bike Composite Index (BCI) consisting of two sub-indices representing bike attractiveness and bike safety, estimated using Bike Kilometers Travelled (BKT) and cyclist-vehicle crash data from 134 traffic analysis zones in Vancouver, Canada. This approach demonstrates the practical application of composite indices in real urban environments, providing actionable insights for local planning decisions.

More recently, Félix et al.~(2022) developed a method for identifying potential locations for cycling infrastructure improvements using open data in Paris, addressing the need for simple and effective methods to support decision-making in bicycle planning. Their approach integrates spatial analysis with accessibility metrics to pinpoint areas requiring infrastructure enhancement.

Advanced computational approaches have also been employed, with Szell et al.~(2022) proposing a framework for generating efficient bike path networks that explicitly considers cyclists' demand distribution and route choices based on safety preferences. This demand-driven design approach represents a significant advancement in evidence-based cycling infrastructure planning.

However, a significant limitation in current theoretical frameworks is the insufficient integration of cycling perceptual factors and subjective safety assessments. Among the 137 indicators identified in bikeability research, only a few relating to air quality were based on cyclist perceptions, highlighting the predominant focus on objective measures. Castro et al.~(2023) emphasise that perceived safety is recognised as a key barrier to cycling, yet its constructs are poorly understood, with most assessments focusing primarily on crash and injury risk rather than broader perceptual dimensions. This gap between objective infrastructure provision and subjective cycling experiences represents a crucial oversight, as perceptual factors significantly influence cycling behaviour and route choices.

Despite these methodological advances, current composite index frameworks still encounter limitations in data integration procedures, objective indicator weighting determination, and computational efficiency for large-scale applications. The insufficient incorporation of perceptual and subjective factors further compounds these challenges, highlighting the ongoing need for more holistic, transparent, and computationally robust methods for comprehensive cycling environment assessment.

\section{Conclusion}\label{conclusion}

Despite the comprehensive literature base, existing cycling environment assessments remain predominantly fragmented, typically addressing only singular dimensions (structural, environmental, or network-based) (Muhs \& Clifton, 2015). This fragmented approach limits the effectiveness of urban cycling environment evaluations, preventing the development of comprehensive and integrative insights crucial for urban planners and policymakers (Giles-Corti et al., 2019). Moreover, few existing frameworks have explicitly addressed the synergistic interactions among different cycling environment factors, particularly the intersection between structural and environmental perception variables.

Addressing these research gaps, this study introduces the Cycling Environment Composite Index (CECI), a comprehensive indicator designed to integrate the critical dimensions of structural rideability, environmental perception, and network performance. The CECI differs from prior studies by explicitly synthesizing diverse but interrelated urban cycling determinants, thereby enhancing evaluation comprehensiveness. While aligning conceptually with the ``15-minute city'' framework---whereby residents can access their daily needs within a 15-minute walk, bicycle or transit ride from their home---the CECI's broader analytical scope ensures greater adaptability and utility for various urban contexts (Moreno et al., 2021). By explicitly integrating indicators such as GVI, air quality, LTS, network connectivity, and multimodal infrastructure, the proposed CECI methodology provides planners with a nuanced understanding of spatial disparities in cycling environment quality, thereby enabling targeted interventions.

The integration of structural, environmental, and network performance indicators into a single composite index provides a robust, practical framework for assessing urban cycling environments comprehensively. Although developed and initially demonstrated in Greater London, the flexible design and theoretical robustness of CECI facilitate its potential adaptation and application to other urban contexts. The proposed index thus contributes to the ongoing development of urban cycling environment evaluation methods, potentially offering useful insights for targeted policy interventions and infrastructure investments to support sustainable urban mobility.

\chapter{Equations and direct quotes}\label{equations-and-direct-quotes}

This section will focus on equations and direct quotes

\section{Equations}\label{equations}

You need to include equations with some LaTeX. The easiest way to do this is to use an online tool such as: \url{https://latex.codecogs.com/eqneditor/editor.php}. It can be a real pain to get these right, but once you've worked it out it will be much easier than dealing with word equation editor.

\begin{equation} 
  p= h\frac{c}{\varrho}
  \label{eq:test}
\end{equation}

To reference this in the text we use: \eqref{eq:test}

You can also test your code in your RMarkdown document using \texttt{\$} e.g.~\[p= h\frac{c}{\varrho}\]

However, this won't generate an equation number and you can't cross reference it. But we can use this logic to define the parameters within the equation e.g.~where \(h\) is Plank's constant, \(6.626 × 10^-34 Js\)

\texttt{\$\$} puts it on a new line, single \texttt{\$} keeps it on the same line (in the text)

\section{Block quotes (or direct quotes)}\label{quotes}

You may wish to quote a large section from a source, to do this use a block quote.

Simply input a \texttt{\textgreater{}} before the text. For example \texttt{\textgreater{}\ This\ is\ a\ quote}.

\begin{quote}
``This is a direct quote''
\end{quote}

You can also provide an attribution at the footer of the quote using \texttt{tufte::quote\_footer()}, either a name or a reference. You'll need to install the \texttt{tufte} package to use this. For example, \texttt{\textgreater{}\ {[}include\ only\ r\ here{]}\ tufte::quote\_footer(\textquotesingle{}-\/-\/-\ Joe\ Martin\textquotesingle{})} or \texttt{\textgreater{}\ {[}include\ only\ r\ here{]}\ tufte::quote\_footer(\textquotesingle{}-\/-\/-\ @xie2015\textquotesingle{})}

Giving\ldots{}

\begin{quote}
\hfill --- Joe Martin
\end{quote}

\begin{quote}
\hfill --- \textcite{xie2015}
\end{quote}

Instead you could just include the reference at the end of the quote, using the same method to reference as we've seen before..e.g. \texttt{\textgreater{}\ "This\ is\ a\ direct\ quote"\ @xie2015,\ Equation\ \textbackslash{}@ref(eq:test)}\ldots giving

\begin{quote}
``This is a direct quote'' \textcite{xie2015} \eqref{eq:test}
\end{quote}

\chapter{Figures, tables, hosting GitBook}\label{figures-tables-hosting-gitbook}

This section is going to focus including figures and creating tables

\begin{figure}
\includegraphics[width=300pt]{general_images/example_flow} \caption{Summary of methdological procedure for (a).... and (b)....}\label{fig:methodsflow}
\end{figure}

\section{Including figures and tables}\label{including-figures-and-tables}

\subsection{Figures}\label{figures}

To include the figure above use the code:

\texttt{knitr::include\_graphics(here::here(\textquotesingle{}general\_images\textquotesingle{},\textquotesingle{}example\_flow.png\textquotesingle{}))}

within a code chunk. In the chunk options you can specify the width and figure captions e.g.~

\texttt{out.width="100pt",\ fig.cap="Summary\ of\ methdological\ procedure\ for\ (a)....\ and\ (b)...."}.

However, if you do show the code with \texttt{echo=TRUE} then you can't specify the \texttt{out.width}.

For making flow diagrams have a look at:

\begin{enumerate}
\def\labelenumi{\arabic{enumi}.}
\tightlist
\item
  Lucidchart
\item
  Draw.io
\end{enumerate}

\subsection{Tables}\label{tables}

For creating tables i'd suggest creating either an excel file or \texttt{.csv} and then reading the data into R and using the \texttt{kabble} package to format it how you wish. The example below is from the abbreviations section

\begin{Shaded}
\begin{Highlighting}[]
\FunctionTok{library}\NormalTok{(tidyverse)}
\FunctionTok{library}\NormalTok{(knitr)}
\FunctionTok{library}\NormalTok{(kableExtra)}
\FunctionTok{library}\NormalTok{(readxl)}
\FunctionTok{library}\NormalTok{(fs)}
\FunctionTok{library}\NormalTok{(here)}

\CommentTok{\#read in data}
\FunctionTok{read\_excel}\NormalTok{(}\FunctionTok{here}\NormalTok{(}\StringTok{"tables"}\NormalTok{, }\StringTok{"abbreviations.xlsx"}\NormalTok{))}\SpecialCharTok{\%\textgreater{}\%}
  \FunctionTok{arrange}\NormalTok{(Term) }\SpecialCharTok{\%\textgreater{}\%} \CommentTok{\# i.e. alphabetical order by Term}
  \CommentTok{\# booktab = T gives us a pretty APA{-}ish table}
\NormalTok{  knitr}\SpecialCharTok{::}\FunctionTok{kable}\NormalTok{(}\AttributeTok{booktabs =} \ConstantTok{TRUE}\NormalTok{)}\SpecialCharTok{\%\textgreater{}\%} 
  \FunctionTok{kable\_styling}\NormalTok{(}\AttributeTok{position =} \StringTok{"center"}\NormalTok{)}\SpecialCharTok{\%\textgreater{}\%}
  \CommentTok{\# any specifc row changes you want}
    \FunctionTok{row\_spec}\NormalTok{(.,}
  \AttributeTok{row=}\DecValTok{0}\NormalTok{,}
  \AttributeTok{bold =} \ConstantTok{TRUE}\NormalTok{)}
\end{Highlighting}
\end{Shaded}

\begin{table}
\centering
\begin{tabular}{ll}
\toprule
\textbf{Term} & \textbf{Abbreviation}\\
\midrule
Digital Elevation Model & DEM\\
Digital Surface Model & DSM\\
Digital Terrain Model & DTM\\
\bottomrule
\end{tabular}
\end{table}

You do loads of things with kabble including adding small visulisations within the table - \href{https://cran.r-project.org/web/packages/kableExtra/vignettes/awesome_table_in_html.html\#Overview}{consult the documentation for more info}.

\textbf{If in doubt, keep it simple}

other useful arguments for tables:

\begin{itemize}
\item
  \texttt{column\_spec(2,\ width\ =\ "9cm")} = set column width
\item
  \texttt{kable(timeline,longtable\ =\ T....}= allow the table to go over multiple pages
\end{itemize}

For example\ldots{}
\newpage

\begin{Shaded}
\begin{Highlighting}[]
\FunctionTok{read\_excel}\NormalTok{(}\FunctionTok{here}\NormalTok{(}\StringTok{"tables"}\NormalTok{, }\StringTok{"policy.xlsx"}\NormalTok{))}\SpecialCharTok{\%\textgreater{}\%}
  \FunctionTok{mutate\_all}\NormalTok{(}\SpecialCharTok{\textasciitilde{}} \FunctionTok{replace\_na}\NormalTok{(.x, }\StringTok{""}\NormalTok{)) }\SpecialCharTok{\%\textgreater{}\%}
  \CommentTok{\# booktab = T gives us a pretty APA{-}ish table}
\NormalTok{  knitr}\SpecialCharTok{::}\FunctionTok{kable}\NormalTok{(}\AttributeTok{longtable =}\NormalTok{ T, }\AttributeTok{booktabs =}\NormalTok{ T, }
               \AttributeTok{caption =} \StringTok{\textquotesingle{}Relevant influential international, metropolitan and local UHI and urban expansion policies, strategies and assessments (with publication date) referred to in this paper. * Denotes documents that lack specific UHI related policy but recognise the value of maintaining vegetation.\textquotesingle{}}\NormalTok{)}\SpecialCharTok{\%\textgreater{}\%} 
  \FunctionTok{kable\_styling}\NormalTok{(}\AttributeTok{position =} \StringTok{"center"}\NormalTok{, }\AttributeTok{full\_width =}\NormalTok{ T)}\SpecialCharTok{\%\textgreater{}\%}
  \CommentTok{\# any specifc row changes you want}
    \FunctionTok{row\_spec}\NormalTok{(.,}
  \AttributeTok{row=}\FunctionTok{c}\NormalTok{(}\DecValTok{0}\NormalTok{,}\DecValTok{1}\NormalTok{,}\DecValTok{8}\NormalTok{, }\DecValTok{18}\NormalTok{),}
  \AttributeTok{bold =} \ConstantTok{TRUE}\NormalTok{)}\SpecialCharTok{\%\textgreater{}\%}
  \FunctionTok{column\_spec}\NormalTok{(}\DecValTok{1}\NormalTok{, }\AttributeTok{width =} \StringTok{"14cm"}\NormalTok{)}\SpecialCharTok{\%\textgreater{}\%}
  \FunctionTok{row\_spec}\NormalTok{(}\FunctionTok{c}\NormalTok{(}\DecValTok{1}\NormalTok{, }\DecValTok{8}\NormalTok{, }\DecValTok{18}\NormalTok{), }\AttributeTok{hline\_after =}\NormalTok{ T)}
\end{Highlighting}
\end{Shaded}

\begin{longtabu} to \linewidth {>{\raggedright\arraybackslash}p{14cm}}
\caption{\label{tab:kable}Relevant influential international, metropolitan and local UHI and urban expansion policies, strategies and assessments (with publication date) referred to in this paper. * Denotes documents that lack specific UHI related policy but recognise the value of maintaining vegetation.}\\
\toprule
\textbf{Policy}\\
\midrule
\textbf{International}\\
\midrule
United Nations The World Cities in 2016 (2016)\\
United Nations New Urban Agenda (2017)\\
ARUP City Resilience Framework (2015)\\
United Nations International Strategy for Disaster Reduction Sendai Framework (2015)\\
\addlinespace
Universal Sustainable Development Goals (2015)\\
Biological Diversity, Cities and Biodiversity Outlook (2012)\\
\textbf{Metropolitan}\\
\midrule
AECOM Australia, Economic Assessment of the Urban Heat Island Effect, Melbourne (2012)\\
The Spatial Development Strategy For Greater London (2017)\\
\addlinespace
Western Australia Planning Commission, Perth and Peel @3.5 million (2015)\\
City of Johannesburg Metropolitan Municipality, Spatial Development Framework 2040 (2016)\\
Western Australian Planning Commission, Directions 2031 and beyond: metropolitan planning beyond the horizon (2010)\\
Western Australian Planning Commission, Development Control Policy 2.3 Public Open Space in Residential Areas (2002)\\
Plan For The Metropolitan Region Perth And Fremantle (1955)\\
\addlinespace
Singapore Government, Open Space Provision (2011)\\
Western Australian Planning Commission, Metropolitan Region Scheme Text (2006)\\
\textbf{Local}\\
\midrule
USA Environmental Protection Agency, Reducing Urban Heat Islands Compendium of Strategies Trees and Vegetation (2008)\\
USA Environmental Protection Agency, Reducing Urban Heat Islands Compendium of Strategies Urban Heat Island Basics (2008)\\
\addlinespace
USA Environmental Protection Agency, Reducing Urban Heat Islands, Compendium of Strategies Heat Island Reduction Activities (2008)\\
City of Stirling, Stirling Urban Forest Community Consultation (2017)\\
City of Fremantle, One Planet Fremantle Strategy 2014/2015 - 2019/2020, 1–12 (2014)\\
Metropolitan Redevelopment Authority, Subiaco Redevelopment Scheme (2013)\\
Metropolitan Redevelopment Authority, Subiaco Redevelopment Scheme 2 (2017)\\
\addlinespace
City of Bayswater, Urban Forest Strategy (2017)\\
City of Perth, Urban Forest Plan 2016-2036 (2006)\\
City of Fremantle, City of Fremantle Urban Forest Plan (2017)\\
City of Fremantle, Annual Budget 2016-17 (2016)\\
City of Wanneroo, Street Tree Policy (2016)*\\
\addlinespace
City of Subiaco, Plant Pathogen Management Plan(2015)*\\
\bottomrule
\end{longtabu}

\textbf{Note} if you see the caption in the \texttt{.pdf} version it goes off the side of the page --- this is the reason why you don't show code in a \texttt{.pdf}. If you ever had to you could just seperate the string into sections and at the start use \texttt{paste("hello","this","is","a","string",\ sep="\ ")}

Now remember to cross reference this table, it would be\ldots Table \texttt{\textbackslash{}@ref(tab:kable)}, giving Table \ref{tab:kable}

\section{Hosting the book}\label{hosting-the-book}

You will need to create a new GitHub repository to host your book online using GitHub pages --- like the example is. GitHub pages takes a load of \texttt{.html} files and makes a website.

To do so you need to set up a few things

\begin{enumerate}
\def\labelenumi{\arabic{enumi}.}
\tightlist
\item
  Go to the \texttt{\_bookdown.yml} file and make sue that that you have this line of code: \texttt{output\_dir:\ docs} (it should be there)
\item
  In the same file make sure your \texttt{book\_filename} doesn't have any spaces use \texttt{-} or \texttt{\_} e.g.~\texttt{CASA-Thesis}
\item
  Go to the \texttt{\_output.yml} file and change the \texttt{edit} argument to \texttt{YOURREPO/edit/main/\%s}, here it's \texttt{https://github.com/andrewmaclachlan/CASA-MSc-thesis/edit/main/\%s}
\item
  Build your book locally, close the preview window
\item
  Save, stage changes, commit and then push to GitHub
\item
  On your GitHub repository \textgreater{} settings \textgreater{} GitHub pages \textgreater{} select the source as main and the folder as docs
\item
  Make sure you build your \texttt{.pdf} and then your \texttt{gitbook} for the latest \texttt{.pdf} to be a download option on the website.
\end{enumerate}

\chapter{Discussion}\label{discussion}

Short introduction to the chapter, reviewing the previous chapter and detailing what this one aims to achieve and build upon.

\section{Research significance}\label{research-significance}

\subsection{Global development goals}\label{global-development-goals}

\subsection{Local policy}\label{local-policy}

\subsection{Academic research}\label{academic-research}

\section{Limitations}\label{limitations}

\section{Transferability}\label{transferability}

\chapter{Conclusion}\label{conclusion-1}

Short introduction to the chapter, reviewing the previous chapter and detailing what this one aims to achieve and build upon.

\section{Recommedations}\label{recommedations}

\begin{enumerate}
\def\labelenumi{\arabic{enumi}.}
\tightlist
\item
  Adapt policy x
\item
  Undertake data informed targeted greening
\item
  Further work into this area
\end{enumerate}

\addcontentsline{toc}{chapter}{Bibliography}
\printbibliography

\chapter*{Appendix A Research log}\label{appendix-a-research-log}
\addcontentsline{toc}{chapter}{Appendix A Research log}

\addtocontents{toc}{\protect\setcounter{tocdepth}{0}}

\section*{subsection}\label{subsection}

\subsection*{sub sub section}\label{sub-sub-section}

\begin{table}
\centering
\begin{tabular}{ll}
\toprule
\textbf{Date} & \textbf{Task}\\
\midrule
31st May 2020 & data search, commenced literature review\\
7th June 2020 & revised literature in the direction of x\\
\bottomrule
\end{tabular}
\end{table}

\chapter*{Appendix B Proposal}\label{appendix-b-proposal}

\addtocontents{toc}{\protect\setcounter{tocdepth}{3}}
\enddocument

\printbibliography

\end{document}
